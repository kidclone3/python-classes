% Options for packages loaded elsewhere
\PassOptionsToPackage{unicode}{hyperref}
\PassOptionsToPackage{hyphens}{url}
%
\documentclass[
]{article}
\usepackage{amsmath,amssymb}
\usepackage{lmodern}
\usepackage{iftex}
\ifPDFTeX
  \usepackage[T1]{fontenc}
  \usepackage[utf8]{inputenc}
  \usepackage{textcomp} % provide euro and other symbols
\else % if luatex or xetex
  \usepackage{unicode-math}
  \defaultfontfeatures{Scale=MatchLowercase}
  \defaultfontfeatures[\rmfamily]{Ligatures=TeX,Scale=1}
\fi
% Use upquote if available, for straight quotes in verbatim environments
\IfFileExists{upquote.sty}{\usepackage{upquote}}{}
\IfFileExists{microtype.sty}{% use microtype if available
  \usepackage[]{microtype}
  \UseMicrotypeSet[protrusion]{basicmath} % disable protrusion for tt fonts
}{}
\makeatletter
\@ifundefined{KOMAClassName}{% if non-KOMA class
  \IfFileExists{parskip.sty}{%
    \usepackage{parskip}
  }{% else
    \setlength{\parindent}{0pt}
    \setlength{\parskip}{6pt plus 2pt minus 1pt}}
}{% if KOMA class
  \KOMAoptions{parskip=half}}
\makeatother
\usepackage{xcolor}
\usepackage[margin=1in]{geometry}
\usepackage{color}
\usepackage{fancyvrb}
\newcommand{\VerbBar}{|}
\newcommand{\VERB}{\Verb[commandchars=\\\{\}]}
\DefineVerbatimEnvironment{Highlighting}{Verbatim}{commandchars=\\\{\}}
% Add ',fontsize=\small' for more characters per line
\usepackage{framed}
\definecolor{shadecolor}{RGB}{42,33,28}
\newenvironment{Shaded}{\begin{snugshade}}{\end{snugshade}}
\newcommand{\AlertTok}[1]{\textcolor[rgb]{1.00,1.00,0.00}{#1}}
\newcommand{\AnnotationTok}[1]{\textcolor[rgb]{0.00,0.40,1.00}{\textbf{\textit{#1}}}}
\newcommand{\AttributeTok}[1]{\textcolor[rgb]{0.74,0.68,0.62}{#1}}
\newcommand{\BaseNTok}[1]{\textcolor[rgb]{0.27,0.67,0.26}{#1}}
\newcommand{\BuiltInTok}[1]{\textcolor[rgb]{0.74,0.68,0.62}{#1}}
\newcommand{\CharTok}[1]{\textcolor[rgb]{0.02,0.61,0.04}{#1}}
\newcommand{\CommentTok}[1]{\textcolor[rgb]{0.00,0.40,1.00}{\textbf{\textit{#1}}}}
\newcommand{\CommentVarTok}[1]{\textcolor[rgb]{0.74,0.68,0.62}{#1}}
\newcommand{\ConstantTok}[1]{\textcolor[rgb]{0.74,0.68,0.62}{#1}}
\newcommand{\ControlFlowTok}[1]{\textcolor[rgb]{0.26,0.66,0.93}{\textbf{#1}}}
\newcommand{\DataTypeTok}[1]{\textcolor[rgb]{0.74,0.68,0.62}{\underline{#1}}}
\newcommand{\DecValTok}[1]{\textcolor[rgb]{0.27,0.67,0.26}{#1}}
\newcommand{\DocumentationTok}[1]{\textcolor[rgb]{0.00,0.40,1.00}{\textit{#1}}}
\newcommand{\ErrorTok}[1]{\textcolor[rgb]{1.00,1.00,0.00}{\textbf{#1}}}
\newcommand{\ExtensionTok}[1]{\textcolor[rgb]{0.74,0.68,0.62}{#1}}
\newcommand{\FloatTok}[1]{\textcolor[rgb]{0.27,0.67,0.26}{#1}}
\newcommand{\FunctionTok}[1]{\textcolor[rgb]{1.00,0.58,0.35}{\textbf{#1}}}
\newcommand{\ImportTok}[1]{\textcolor[rgb]{0.74,0.68,0.62}{#1}}
\newcommand{\InformationTok}[1]{\textcolor[rgb]{0.00,0.40,1.00}{\textbf{\textit{#1}}}}
\newcommand{\KeywordTok}[1]{\textcolor[rgb]{0.26,0.66,0.93}{\textbf{#1}}}
\newcommand{\NormalTok}[1]{\textcolor[rgb]{0.74,0.68,0.62}{#1}}
\newcommand{\OperatorTok}[1]{\textcolor[rgb]{0.74,0.68,0.62}{#1}}
\newcommand{\OtherTok}[1]{\textcolor[rgb]{0.74,0.68,0.62}{#1}}
\newcommand{\PreprocessorTok}[1]{\textcolor[rgb]{0.74,0.68,0.62}{\textbf{#1}}}
\newcommand{\RegionMarkerTok}[1]{\textcolor[rgb]{0.74,0.68,0.62}{#1}}
\newcommand{\SpecialCharTok}[1]{\textcolor[rgb]{0.02,0.61,0.04}{#1}}
\newcommand{\SpecialStringTok}[1]{\textcolor[rgb]{0.02,0.61,0.04}{#1}}
\newcommand{\StringTok}[1]{\textcolor[rgb]{0.02,0.61,0.04}{#1}}
\newcommand{\VariableTok}[1]{\textcolor[rgb]{0.74,0.68,0.62}{#1}}
\newcommand{\VerbatimStringTok}[1]{\textcolor[rgb]{0.02,0.61,0.04}{#1}}
\newcommand{\WarningTok}[1]{\textcolor[rgb]{1.00,1.00,0.00}{\textbf{#1}}}
\usepackage{longtable,booktabs,array}
\usepackage{calc} % for calculating minipage widths
% Correct order of tables after \paragraph or \subparagraph
\usepackage{etoolbox}
\makeatletter
\patchcmd\longtable{\par}{\if@noskipsec\mbox{}\fi\par}{}{}
\makeatother
% Allow footnotes in longtable head/foot
\IfFileExists{footnotehyper.sty}{\usepackage{footnotehyper}}{\usepackage{footnote}}
\makesavenoteenv{longtable}
\usepackage{graphicx}
\makeatletter
\def\maxwidth{\ifdim\Gin@nat@width>\linewidth\linewidth\else\Gin@nat@width\fi}
\def\maxheight{\ifdim\Gin@nat@height>\textheight\textheight\else\Gin@nat@height\fi}
\makeatother
% Scale images if necessary, so that they will not overflow the page
% margins by default, and it is still possible to overwrite the defaults
% using explicit options in \includegraphics[width, height, ...]{}
\setkeys{Gin}{width=\maxwidth,height=\maxheight,keepaspectratio}
% Set default figure placement to htbp
\makeatletter
\def\fps@figure{htbp}
\makeatother
\setlength{\emergencystretch}{3em} % prevent overfull lines
\providecommand{\tightlist}{%
  \setlength{\itemsep}{0pt}\setlength{\parskip}{0pt}}
\setcounter{secnumdepth}{-\maxdimen} % remove section numbering
\ifLuaTeX
  \usepackage{selnolig}  % disable illegal ligatures
\fi
\IfFileExists{bookmark.sty}{\usepackage{bookmark}}{\usepackage{hyperref}}
\IfFileExists{xurl.sty}{\usepackage{xurl}}{} % add URL line breaks if available
\urlstyle{same} % disable monospaced font for URLs
\hypersetup{
  pdftitle={Homework week 2},
  pdfauthor={Bùi Khánh Duy},
  hidelinks,
  pdfcreator={LaTeX via pandoc}}

\title{Homework week 2}
\author{Bùi Khánh Duy}
\date{2023-03-08}

\begin{document}
\maketitle

\hypertarget{homework}{%
\subsection{Homework}\label{homework}}

\hypertarget{list-all-possible-simple-random-samples-of-size-n-2-that-can-be-selected-from-the-population-0-1-2-3-4.-calculate-sigma-2-for-the-population-and-vbary-for-the-sample.}{%
\subsubsection{\texorpdfstring{4.1 List all possible simple random
samples of size n = 2 that can be selected from the population \{0, 1,
2, 3, 4\}. Calculate \(\sigma^ 2\) for the population and \(V(\bar{y})\)
for the
sample.}{4.1 List all possible simple random samples of size n = 2 that can be selected from the population \{0, 1, 2, 3, 4\}. Calculate \textbackslash sigma\^{} 2 for the population and V(\textbackslash bar\{y\}) for the sample.}}\label{list-all-possible-simple-random-samples-of-size-n-2-that-can-be-selected-from-the-population-0-1-2-3-4.-calculate-sigma-2-for-the-population-and-vbary-for-the-sample.}}

\begin{Shaded}
\begin{Highlighting}[]
\NormalTok{dataX }\OtherTok{\textless{}{-}} \FunctionTok{c}\NormalTok{(}\DecValTok{0}\NormalTok{,}\DecValTok{1}\NormalTok{,}\DecValTok{2}\NormalTok{,}\DecValTok{3}\NormalTok{,}\DecValTok{4}\NormalTok{)}
\NormalTok{N }\OtherTok{\textless{}{-}} \FunctionTok{length}\NormalTok{(dataX)}
\NormalTok{n }\OtherTok{\textless{}{-}} \DecValTok{2}
\NormalTok{x }\OtherTok{\textless{}{-}} \FunctionTok{t}\NormalTok{(}\FunctionTok{combn}\NormalTok{(dataX, n))}
\end{Highlighting}
\end{Shaded}

\begin{verbatim}
##       [,1] [,2]
##  [1,]    0    1
##  [2,]    0    2
##  [3,]    0    3
##  [4,]    0    4
##  [5,]    1    2
##  [6,]    1    3
##  [7,]    1    4
##  [8,]    2    3
##  [9,]    2    4
## [10,]    3    4
\end{verbatim}

\begin{Shaded}
\begin{Highlighting}[]
\NormalTok{Sample }\OtherTok{\textless{}{-}} \FunctionTok{as.vector.data.frame}\NormalTok{(}\FunctionTok{apply}\NormalTok{(x, }\AttributeTok{MARGIN =} \DecValTok{1}\NormalTok{, }\ControlFlowTok{function}\NormalTok{(x) \{}
  \FunctionTok{paste}\NormalTok{(}\FunctionTok{sprintf}\NormalTok{(}\StringTok{"\{\%d,\%d\}"}\NormalTok{,x[}\DecValTok{1}\NormalTok{], x[}\DecValTok{2}\NormalTok{]))}
\NormalTok{\}))}
\NormalTok{y\_bar }\OtherTok{\textless{}{-}} \FunctionTok{as.vector.data.frame}\NormalTok{(}\FunctionTok{rowMeans}\NormalTok{(x))}

\NormalTok{data2 }\OtherTok{\textless{}{-}} \FunctionTok{data.frame}\NormalTok{(Sample, y\_bar)}
\end{Highlighting}
\end{Shaded}

\begin{longtable}[]{@{}lr@{}}
\toprule()
Sample & y\_bar \\
\midrule()
\endhead
\{0,1\} & 0.5 \\
\{0,2\} & 1.0 \\
\{0,3\} & 1.5 \\
\{0,4\} & 2.0 \\
\{1,2\} & 1.5 \\
\{1,3\} & 2.0 \\
\bottomrule()
\end{longtable}

and \[\sigma^2 = V(y) = E(y-\mu)^2 \\
\text{with } \mu = E(\bar{y})
\]

\begin{Shaded}
\begin{Highlighting}[]
\NormalTok{y\_bar.mean }\OtherTok{\textless{}{-}} \FunctionTok{mean}\NormalTok{(y\_bar)}
\NormalTok{y\_bar.mean}
\end{Highlighting}
\end{Shaded}

\begin{verbatim}
## [1] 2
\end{verbatim}

\begin{Shaded}
\begin{Highlighting}[]
\NormalTok{sigma }\OtherTok{=} \FunctionTok{mean}\NormalTok{((x }\SpecialCharTok{{-}}\NormalTok{ y\_bar.mean)}\SpecialCharTok{\^{}}\DecValTok{2}\NormalTok{)}
\FunctionTok{print}\NormalTok{(}\FunctionTok{paste}\NormalTok{(}\StringTok{"sigma\^{}2 = "}\NormalTok{,sigma))}
\end{Highlighting}
\end{Shaded}

\begin{verbatim}
## [1] "sigma^2 =  2"
\end{verbatim}

\[V(\bar{y}) = E(\bar{y}-\mu)^2 \]

\begin{Shaded}
\begin{Highlighting}[]
\NormalTok{v\_y\_bar }\OtherTok{\textless{}{-}} \FunctionTok{mean}\NormalTok{((y\_bar }\SpecialCharTok{{-}}\NormalTok{ y\_bar.mean)}\SpecialCharTok{\^{}}\DecValTok{2}\NormalTok{)}
\NormalTok{v\_y\_bar}
\end{Highlighting}
\end{Shaded}

\begin{verbatim}
## [1] 0.75
\end{verbatim}

\begin{Shaded}
\begin{Highlighting}[]
\NormalTok{sigma}\SpecialCharTok{/}\NormalTok{n}\SpecialCharTok{*}\NormalTok{(N}\SpecialCharTok{{-}}\NormalTok{n)}\SpecialCharTok{/}\NormalTok{(N}\DecValTok{{-}1}\NormalTok{)}
\end{Highlighting}
\end{Shaded}

\begin{verbatim}
## [1] 0.75
\end{verbatim}

\[\Rightarrow V(\bar{y}) = \frac{\sigma^2}n \bigg(\frac{N-n}{N-1}\bigg)\]

\hypertarget{for-the-simple-random-samples-generated-in-exercise-4.1-calculate-s2-for-each-sample.-show-numerically-that}{%
\subsubsection{\texorpdfstring{4.2 For the simple random samples
generated in Exercise 4.1, calculate \(s^2\) for each sample. Show
numerically
that}{4.2 For the simple random samples generated in Exercise 4.1, calculate s\^{}2 for each sample. Show numerically that}}\label{for-the-simple-random-samples-generated-in-exercise-4.1-calculate-s2-for-each-sample.-show-numerically-that}}

\[E(s^2) = \frac{N}{N-1}\sigma^2\]

\begin{Shaded}
\begin{Highlighting}[]
\NormalTok{s\_2 }\OtherTok{\textless{}{-}} \FunctionTok{as.vector.data.frame}\NormalTok{(}\FunctionTok{apply}\NormalTok{(x, }\AttributeTok{MARGIN=}\DecValTok{1}\NormalTok{, }\ControlFlowTok{function}\NormalTok{(x) \{}
  \FunctionTok{var}\NormalTok{(x)}
\NormalTok{\}))}
\NormalTok{data3 }\OtherTok{\textless{}{-}} \FunctionTok{data.frame}\NormalTok{(data2, s\_2)}
\end{Highlighting}
\end{Shaded}

\begin{longtable}[]{@{}lrr@{}}
\toprule()
Sample & y\_bar & s\_2 \\
\midrule()
\endhead
\{0,1\} & 0.5 & 0.5 \\
\{0,2\} & 1.0 & 2.0 \\
\{0,3\} & 1.5 & 4.5 \\
\{0,4\} & 2.0 & 8.0 \\
\{1,2\} & 1.5 & 0.5 \\
\{1,3\} & 2.0 & 2.0 \\
\{1,4\} & 2.5 & 4.5 \\
\{2,3\} & 2.5 & 0.5 \\
\{2,4\} & 3.0 & 2.0 \\
\{3,4\} & 3.5 & 0.5 \\
\bottomrule()
\end{longtable}

\begin{Shaded}
\begin{Highlighting}[]
\NormalTok{s\_2.mean }\OtherTok{\textless{}{-}} \FunctionTok{mean}\NormalTok{(s\_2)}
\NormalTok{s\_2.mean}
\end{Highlighting}
\end{Shaded}

\begin{verbatim}
## [1] 2.5
\end{verbatim}

\begin{Shaded}
\begin{Highlighting}[]
\NormalTok{N}\SpecialCharTok{/}\NormalTok{(N}\DecValTok{{-}1}\NormalTok{)}\SpecialCharTok{*}\NormalTok{sigma}
\end{Highlighting}
\end{Shaded}

\begin{verbatim}
## [1] 2.5
\end{verbatim}

\[\Rightarrow E(s^2) = \frac{N}{N-1}\sigma^2\] That all!

\hypertarget{suppose-you-want-to-estimate-the-number-of-weed-clusters-of-a-certain-type-in-a-field.-what-is-the-population-and-what-would-you-use-for-sampling-units-how-would-you-construct-a-frame-how-would-you-select-a-simple-random-sample-if-a-sampling-unit-is-an-area-such-as-a-square-yard-does-the-size-chosen-for-a-sampling-unit-affect-the-accuracy-of-the-results-what-considerations-go-into-our-choice-of-size-of-sampling-unit}{%
\subsubsection{4.3 Suppose you want to estimate the number of weed
clusters of a certain type in a field. What is the population, and what
would you use for sampling units? How would you construct a frame? How
would you select a simple random sample? If a sampling unit is an area
such as a square yard, does the size chosen for a sampling unit affect
the accuracy of the results? What considerations go into our choice of
size of sampling
unit?}\label{suppose-you-want-to-estimate-the-number-of-weed-clusters-of-a-certain-type-in-a-field.-what-is-the-population-and-what-would-you-use-for-sampling-units-how-would-you-construct-a-frame-how-would-you-select-a-simple-random-sample-if-a-sampling-unit-is-an-area-such-as-a-square-yard-does-the-size-chosen-for-a-sampling-unit-affect-the-accuracy-of-the-results-what-considerations-go-into-our-choice-of-size-of-sampling-unit}}

\begin{itemize}
\tightlist
\item
  Tổng thể (population): Tất cả các cụm cỏ trên cánh đồng
\item
  Đơn vị mẫu: Sử dụng đơn vị diện tích: mẫu, sào, mét vuông, v.v
\item
  Khung mẫu: Tiến hành đo đạc toàn bộ cánh đồng để biết diện tích của
  tổng thể là bao nhiêu, từ đấy tiến hành chia nhỏ và đánh số để chọn
  mẫu
\item
  Chọn mẫu ngẫu nhiên đơn giản: Tiến hành chọn ngẫu nhiên theo số thứ tự
  được đánh.
\item
  Nếu kích thước chọn mẫu là 1 thước vuông (square yard) =\textgreater{}
  đây là đơn vị phổ biến thường được dùng để đo đạc ruộng đất, và nó
  cũng đủ lớn để tiến hành nghiên cứu. Việc đơn vị mẫu lớn hơn có thể
  hiệu quả hơn trong khảo sát, nhưng khi đơn vị mẫu nhỏ hơn lại có thể
  cung cấp kết quả chính xác hơn. Cần phải cân nhắc giữa hiệu quả và độ
  chính xác, và kích thước đơn vị mẫu nên được chọn cẩn thận. Ở bài toán
  ruộng đất thì có thể ưu tiên độ hiệu quả hơn.
\item
  Các yếu tố cần được xem xét khi lựa chọn kích thước của đơn vị mẫu:

  \begin{enumerate}
  \def\labelenumi{\arabic{enumi}.}
  \tightlist
  \item
    Tính đồng nhất: Nếu tổng thể là đồng nhất, có thể sử dụng các đơn vị
    mẫu nhỏ vì đã có thể bao quát được sự biến động của đơn vị quan sát.
  \item
    Tính đa dạng: Nếu tổng thể đa dạng, có thể cần sử dụng các đơn vị
    mẫu lớn hơn.
  \item
    Độ chính xác mẫu: Các đơn vị mẫu nhỏ hơn có thể cung cấp độ chính
    xác cao hơn, nhưng có thể đòi hỏi nhiều tài nguyên và thời gian để
    thực hiện chọn mẫu và phân tích.
  \item
    Chi phí: Các đơn vị mẫu lớn hơn có thể hiệu quả về chi phí hơn, vì
    cần ít công sức để lấy mẫu và phân tích hơn, nhưng có thể dẫn đến độ
    chính xác không cao.
  \item
    Phương pháp lấy mẫu: Lựa chọn kích thước đơn vị mẫu cũng có thể bị
    ảnh hưởng bởi phương pháp lấy mẫu. Ví dụ, lấy mẫu phân tầng có thể
    đòi hỏi đơn vị mẫu nhỏ hơn so với phương pháp lấy mẫu ngẫu nhiên đơn
    giản.
  \end{enumerate}
\end{itemize}

\hypertarget{in-which-of-the-following-situations-can-you-reasonably-generalize-from-the-sample-to-the-population}{%
\subsubsection{4.4 In which of the following situations can you
reasonably generalize from the sample to the
population?}\label{in-which-of-the-following-situations-can-you-reasonably-generalize-from-the-sample-to-the-population}}

\hypertarget{a.-you-use-your-statistics-class-to-get-an-estimate-of-the-percentage-of-students-in-your-school-who-study-at-least-two-hours-a-night.}{%
\paragraph{a. You use your statistics class to get an estimate of the
percentage of students in your school who study at least two hours a
night.}\label{a.-you-use-your-statistics-class-to-get-an-estimate-of-the-percentage-of-students-in-your-school-who-study-at-least-two-hours-a-night.}}

Có thể, bởi vì việc chọn mẫu là khả thi, có thể thực hiện và có đủ lớn
để kết luận cho tổng thể.

\hypertarget{b.-you-use-the-average-annual-income-of-the-ambassadors-to-the-united-nations-to-get-an-estimate-of-average-per-capita-income-for-the-world-as-a-whole.}{%
\paragraph{b. You use the average annual income of the ambassadors to
the United Nations to get an estimate of average per-capita income for
the world as a
whole.}\label{b.-you-use-the-average-annual-income-of-the-ambassadors-to-the-united-nations-to-get-an-estimate-of-average-per-capita-income-for-the-world-as-a-whole.}}

Không thể. Vì mẫu chọn được chỉ là 1 quốc gia Mỹ trong khi trên thế giới
có gần 300 quốc gia.

\hypertarget{c.-in-1996-a-gallup-poll-sampled-235-u.s.-residents-ages-18-to-29-to-estimate-the-per--centage-of-all-u.s.-residents-ages-18-to-29-who-favored-cuts-in-social-spending.}{%
\paragraph{c.~In 1996, a Gallup poll sampled 235 U.S. residents ages 18
to 29, to estimate the per- centage of all U.S. residents ages 18 to 29
who favored cuts in social
spending.}\label{c.-in-1996-a-gallup-poll-sampled-235-u.s.-residents-ages-18-to-29-to-estimate-the-per--centage-of-all-u.s.-residents-ages-18-to-29-who-favored-cuts-in-social-spending.}}

Không biết, vì không đủ dữ kiện về cách chọn mẫu. Nhưng nếu dựa trên tỉ
lệ cho tổng thể là tất cả cư dẫn Mỹ từ 18 đến 29 tuổi thì là không đủ.

\hypertarget{describe-the-type-of-sample-selection-bias-that-would-result-from-each-of-these-sampling-methods.}{%
\subsubsection{4.5 Describe the type of sample selection bias that would
result from each of these sampling
methods.}\label{describe-the-type-of-sample-selection-bias-that-would-result-from-each-of-these-sampling-methods.}}

\hypertarget{a.-a-student-wants-to-determine-the-average-size-of-farms-in-a-county-in-iowa.-he-drops-some-rice-randomly-on-a-map-of-the-county-and-uses-the-farms-hit-by-grains-of-rice-as-the-sample.}{%
\paragraph{a. A student wants to determine the average size of farms in
a county in Iowa. He drops some rice randomly on a map of the county and
uses the farms hit by grains of rice as the
sample.}\label{a.-a-student-wants-to-determine-the-average-size-of-farms-in-a-county-in-iowa.-he-drops-some-rice-randomly-on-a-map-of-the-county-and-uses-the-farms-hit-by-grains-of-rice-as-the-sample.}}

Sai lệch chọn mẫu do khung chưa đủ. Vì cách chọn này không theo một hệ
thống lấy mẫu hay khung mẫu cụ thể nào nên rất dễ dẫn đến việc bỏ sót
mẫu.

\hypertarget{b.-in-a-study-about-whether-valedictorians-succeed-big-in-life-a-professor-traveled-across-illinois-attending-high-school-graduations-and-selecting-81-students-to-participate.-.-.-.-he-picked-students-from-the-most-diverse-communities-possible-from-little-rural-schools-to-rich-suburban-schools-near-chicago-to-city-schools.-source-michael-ryan-do-valedictorians-succeed-big-in-life-parade-magazine-may-17-1998-pages-1415.}{%
\paragraph{b. In a study about whether valedictorians ``succeed big in
life,'' a professor ``traveled across Illinois, attending high school
graduations and selecting 81 students to participate. . . . He picked
students from the most diverse communities possible, from little rural
schools to rich suburban schools near Chicago to city schools.'' Source:
Michael Ryan, ``Do Valedictorians Succeed Big in Life?'' Parade
Magazine, May 17, 1998, pages
14--15.}\label{b.-in-a-study-about-whether-valedictorians-succeed-big-in-life-a-professor-traveled-across-illinois-attending-high-school-graduations-and-selecting-81-students-to-participate.-.-.-.-he-picked-students-from-the-most-diverse-communities-possible-from-little-rural-schools-to-rich-suburban-schools-near-chicago-to-city-schools.-source-michael-ryan-do-valedictorians-succeed-big-in-life-parade-magazine-may-17-1998-pages-1415.}}

Sai lệch chọn mẫu do mẫu không ngẫu nhiên. Giáo sư đã lựa chọn học sinh
từ các đa dạng các nhóm, điều này có thể đưa ra sai lệch vào nghiên cứu.
Ví dụ, học sinh từ một số nhóm có thể có cơ hội giáo dục hoặc hoàn cảnh
gia đình khác nhau, ảnh hưởng đến việc thành công của họ trong tương
lai.

\hypertarget{c.-to-estimate-the-percentage-of-students-who-passed-the-first-advanced-placement-statistics-exam-a-teacher-on-an-internet-discussion-list-for-teachers-of-ap-statistics-asked-teachers-on-the-list-to-report-to-him-how-many-of-their-students-took-the-test-and-how-many-passed.}{%
\paragraph{c.~To estimate the percentage of students who passed the
first Advanced Placement Statistics exam, a teacher on an Internet
discussion list for teachers of AP Statistics asked teachers on the list
to report to him how many of their students took the test and how many
passed.}\label{c.-to-estimate-the-percentage-of-students-who-passed-the-first-advanced-placement-statistics-exam-a-teacher-on-an-internet-discussion-list-for-teachers-of-ap-statistics-asked-teachers-on-the-list-to-report-to-him-how-many-of-their-students-took-the-test-and-how-many-passed.}}

Sai lệch chọn mẫu do tự chọn. Việc các giáo viên có thể chọn trả lời
hoặc không / trả lời đúng hoặc sai làm ảnh hưởng đến kết quả chọn mẫu.

\hypertarget{d.-to-find-the-average-length-of-string-in-a-bag-a-student-reaches-in-mixes-up-the-strings-selects-one-mixes-them-up-again-selects-another-and-so-on.}{%
\paragraph{d.~To find the average length of string in a bag, a student
reaches in, mixes up the strings, selects one, mixes them up again,
selects another, and so
on.}\label{d.-to-find-the-average-length-of-string-in-a-bag-a-student-reaches-in-mixes-up-the-strings-selects-one-mixes-them-up-again-selects-another-and-so-on.}}

Sai lệch chọn mẫu do lấy mẫu theo hệ thống. Mỗi lần lấy 1 sợi dây ra
khỏi túi, xác suất của các sợi dây đã bị thay đổi, dẫn đến kết quả sai
trong việc tính mẫu.

\hypertarget{e.-in-1984-ann-landers-conducted-a-poll-on-the-marital-happiness-of-women-by-asking-women-to-write-to-her.}{%
\paragraph{e. In 1984, Ann Landers conducted a poll on the marital
happiness of women by asking women to write to
her.}\label{e.-in-1984-ann-landers-conducted-a-poll-on-the-marital-happiness-of-women-by-asking-women-to-write-to-her.}}

Sai lệch chọn mẫu do phản hồi tự nguyện. Chỉ những phụ nữ quan tâm đến
việc tham gia cuộc khảo sát mới viết thư cho Ann Landers, điều này có
thể dẫn đến sự đại diện quá nhiều của phụ nữ có chung quan điểm về hạnh
phúc hôn nhân.

\end{document}
