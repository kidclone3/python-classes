% Options for packages loaded elsewhere
\PassOptionsToPackage{unicode}{hyperref}
\PassOptionsToPackage{hyphens}{url}
%
\documentclass[
]{article}
\usepackage{amsmath,amssymb}
\usepackage{lmodern}
\usepackage{iftex}
\ifPDFTeX
  \usepackage[T1]{fontenc}
  \usepackage[utf8]{inputenc}
  \usepackage{textcomp} % provide euro and other symbols
\else % if luatex or xetex
  \usepackage{unicode-math}
  \defaultfontfeatures{Scale=MatchLowercase}
  \defaultfontfeatures[\rmfamily]{Ligatures=TeX,Scale=1}
\fi
% Use upquote if available, for straight quotes in verbatim environments
\IfFileExists{upquote.sty}{\usepackage{upquote}}{}
\IfFileExists{microtype.sty}{% use microtype if available
  \usepackage[]{microtype}
  \UseMicrotypeSet[protrusion]{basicmath} % disable protrusion for tt fonts
}{}
\makeatletter
\@ifundefined{KOMAClassName}{% if non-KOMA class
  \IfFileExists{parskip.sty}{%
    \usepackage{parskip}
  }{% else
    \setlength{\parindent}{0pt}
    \setlength{\parskip}{6pt plus 2pt minus 1pt}}
}{% if KOMA class
  \KOMAoptions{parskip=half}}
\makeatother
\usepackage{xcolor}
\usepackage[margin=1in]{geometry}
\usepackage{color}
\usepackage{fancyvrb}
\newcommand{\VerbBar}{|}
\newcommand{\VERB}{\Verb[commandchars=\\\{\}]}
\DefineVerbatimEnvironment{Highlighting}{Verbatim}{commandchars=\\\{\}}
% Add ',fontsize=\small' for more characters per line
\usepackage{framed}
\definecolor{shadecolor}{RGB}{248,248,248}
\newenvironment{Shaded}{\begin{snugshade}}{\end{snugshade}}
\newcommand{\AlertTok}[1]{\textcolor[rgb]{0.94,0.16,0.16}{#1}}
\newcommand{\AnnotationTok}[1]{\textcolor[rgb]{0.56,0.35,0.01}{\textbf{\textit{#1}}}}
\newcommand{\AttributeTok}[1]{\textcolor[rgb]{0.77,0.63,0.00}{#1}}
\newcommand{\BaseNTok}[1]{\textcolor[rgb]{0.00,0.00,0.81}{#1}}
\newcommand{\BuiltInTok}[1]{#1}
\newcommand{\CharTok}[1]{\textcolor[rgb]{0.31,0.60,0.02}{#1}}
\newcommand{\CommentTok}[1]{\textcolor[rgb]{0.56,0.35,0.01}{\textit{#1}}}
\newcommand{\CommentVarTok}[1]{\textcolor[rgb]{0.56,0.35,0.01}{\textbf{\textit{#1}}}}
\newcommand{\ConstantTok}[1]{\textcolor[rgb]{0.00,0.00,0.00}{#1}}
\newcommand{\ControlFlowTok}[1]{\textcolor[rgb]{0.13,0.29,0.53}{\textbf{#1}}}
\newcommand{\DataTypeTok}[1]{\textcolor[rgb]{0.13,0.29,0.53}{#1}}
\newcommand{\DecValTok}[1]{\textcolor[rgb]{0.00,0.00,0.81}{#1}}
\newcommand{\DocumentationTok}[1]{\textcolor[rgb]{0.56,0.35,0.01}{\textbf{\textit{#1}}}}
\newcommand{\ErrorTok}[1]{\textcolor[rgb]{0.64,0.00,0.00}{\textbf{#1}}}
\newcommand{\ExtensionTok}[1]{#1}
\newcommand{\FloatTok}[1]{\textcolor[rgb]{0.00,0.00,0.81}{#1}}
\newcommand{\FunctionTok}[1]{\textcolor[rgb]{0.00,0.00,0.00}{#1}}
\newcommand{\ImportTok}[1]{#1}
\newcommand{\InformationTok}[1]{\textcolor[rgb]{0.56,0.35,0.01}{\textbf{\textit{#1}}}}
\newcommand{\KeywordTok}[1]{\textcolor[rgb]{0.13,0.29,0.53}{\textbf{#1}}}
\newcommand{\NormalTok}[1]{#1}
\newcommand{\OperatorTok}[1]{\textcolor[rgb]{0.81,0.36,0.00}{\textbf{#1}}}
\newcommand{\OtherTok}[1]{\textcolor[rgb]{0.56,0.35,0.01}{#1}}
\newcommand{\PreprocessorTok}[1]{\textcolor[rgb]{0.56,0.35,0.01}{\textit{#1}}}
\newcommand{\RegionMarkerTok}[1]{#1}
\newcommand{\SpecialCharTok}[1]{\textcolor[rgb]{0.00,0.00,0.00}{#1}}
\newcommand{\SpecialStringTok}[1]{\textcolor[rgb]{0.31,0.60,0.02}{#1}}
\newcommand{\StringTok}[1]{\textcolor[rgb]{0.31,0.60,0.02}{#1}}
\newcommand{\VariableTok}[1]{\textcolor[rgb]{0.00,0.00,0.00}{#1}}
\newcommand{\VerbatimStringTok}[1]{\textcolor[rgb]{0.31,0.60,0.02}{#1}}
\newcommand{\WarningTok}[1]{\textcolor[rgb]{0.56,0.35,0.01}{\textbf{\textit{#1}}}}
\usepackage{graphicx}
\makeatletter
\def\maxwidth{\ifdim\Gin@nat@width>\linewidth\linewidth\else\Gin@nat@width\fi}
\def\maxheight{\ifdim\Gin@nat@height>\textheight\textheight\else\Gin@nat@height\fi}
\makeatother
% Scale images if necessary, so that they will not overflow the page
% margins by default, and it is still possible to overwrite the defaults
% using explicit options in \includegraphics[width, height, ...]{}
\setkeys{Gin}{width=\maxwidth,height=\maxheight,keepaspectratio}
% Set default figure placement to htbp
\makeatletter
\def\fps@figure{htbp}
\makeatother
\setlength{\emergencystretch}{3em} % prevent overfull lines
\providecommand{\tightlist}{%
  \setlength{\itemsep}{0pt}\setlength{\parskip}{0pt}}
\setcounter{secnumdepth}{-\maxdimen} % remove section numbering
\ifLuaTeX
  \usepackage{selnolig}  % disable illegal ligatures
\fi
\IfFileExists{bookmark.sty}{\usepackage{bookmark}}{\usepackage{hyperref}}
\IfFileExists{xurl.sty}{\usepackage{xurl}}{} % add URL line breaks if available
\urlstyle{same} % disable monospaced font for URLs
\hypersetup{
  pdftitle={Homework 2},
  pdfauthor={Bùi Khánh Duy},
  hidelinks,
  pdfcreator={LaTeX via pandoc}}

\title{Homework 2}
\author{Bùi Khánh Duy}
\date{2023-03-12}

\begin{document}
\maketitle

\hypertarget{section}{%
\subsection{4.12}\label{section}}

Giả sử \(p\) là tỷ lệ số lượng thanh niên trong khoảng 18-19 đang đi
học, tức \(p=0.6\)

Sử dụng công thức Bernoully để chọn k người trong 40 người lấy ngẫu
nhiên

\[
P(k) = C^k_{40}.p^k.(1-p)^{40-k}
\]

\begin{Shaded}
\begin{Highlighting}[]
\NormalTok{p }\OtherTok{=} \FloatTok{0.6}

\NormalTok{k }\OtherTok{=} \DecValTok{32}

\NormalTok{p\_k }\OtherTok{=} \FunctionTok{choose}\NormalTok{(}\DecValTok{40}\NormalTok{, k) }\SpecialCharTok{*}\NormalTok{ (p}\SpecialCharTok{\^{}}\NormalTok{k) }\SpecialCharTok{*}\NormalTok{ ((}\DecValTok{1}\SpecialCharTok{{-}}\NormalTok{p) }\SpecialCharTok{\^{}}\NormalTok{ (}\DecValTok{40}\SpecialCharTok{{-}}\NormalTok{k))}

\NormalTok{p\_k}
\end{Highlighting}
\end{Shaded}

\begin{verbatim}
## [1] 0.004011185
\end{verbatim}

Vì \(p_k\) rất bé nên khả năng lấy đc 32 người đi học từ việc chọn 40
người ngẫu nhiên là ko hợp lý

\hypertarget{section-1}{%
\subsection{4.13}\label{section-1}}

\hypertarget{section-2}{%
\subsection{4.14}\label{section-2}}

\begin{Shaded}
\begin{Highlighting}[]
\NormalTok{n }\OtherTok{=} \DecValTok{30}

\NormalTok{N }\OtherTok{=} \DecValTok{300}

\NormalTok{sigma }\OtherTok{=} \DecValTok{25}

\NormalTok{y\_bar }\OtherTok{=}\NormalTok{ sigma }\SpecialCharTok{/}\NormalTok{ n}

\NormalTok{B }\OtherTok{=} \DecValTok{2}\SpecialCharTok{*}\FunctionTok{sqrt}\NormalTok{(y\_bar}\SpecialCharTok{*}\NormalTok{(}\DecValTok{1}\SpecialCharTok{{-}}\NormalTok{y\_bar)}\SpecialCharTok{/}\NormalTok{n)}

\FunctionTok{print}\NormalTok{(}\FunctionTok{sprintf}\NormalTok{(}\StringTok{"y\_bar = \%f, B = \%f "}\NormalTok{ , y\_bar, B))}
\end{Highlighting}
\end{Shaded}

\begin{verbatim}
## [1] "y_bar = 0.833333, B = 0.136083 "
\end{verbatim}

\hypertarget{section-3}{%
\subsection{4.15}\label{section-3}}

\begin{Shaded}
\begin{Highlighting}[]
\NormalTok{B }\OtherTok{=} \FloatTok{0.05}

\NormalTok{n\_0 }\OtherTok{\textless{}{-}}\NormalTok{ N}\SpecialCharTok{*}\NormalTok{(y\_bar}\SpecialCharTok{*}\NormalTok{(}\DecValTok{1}\SpecialCharTok{{-}}\NormalTok{y\_bar))}\SpecialCharTok{/}\NormalTok{((N}\DecValTok{{-}1}\NormalTok{) }\SpecialCharTok{*}\NormalTok{B}\SpecialCharTok{\^{}}\DecValTok{2}\SpecialCharTok{/}\DecValTok{4} \SpecialCharTok{+}\NormalTok{ y\_bar}\SpecialCharTok{*}\NormalTok{(}\DecValTok{1}\SpecialCharTok{{-}}\NormalTok{y\_bar))}

\NormalTok{n\_0}
\end{Highlighting}
\end{Shaded}

\begin{verbatim}
## [1] 127.9045
\end{verbatim}

Vậy cần \(n \ge \lceil n_0 \rceil \approx n > 128\)

\hypertarget{section-4}{%
\subsection{4.16}\label{section-4}}

\begin{Shaded}
\begin{Highlighting}[]
\NormalTok{N }\OtherTok{=} \DecValTok{10000}
\NormalTok{n }\OtherTok{=} \DecValTok{100}
\NormalTok{y\_bar }\OtherTok{=} \FloatTok{12.5}
\NormalTok{s\_2 }\OtherTok{=} \DecValTok{1252}

\NormalTok{mu }\OtherTok{=}\NormalTok{ y\_bar}

\NormalTok{B }\OtherTok{=} \DecValTok{2}\SpecialCharTok{*}\FunctionTok{sqrt}\NormalTok{((}\DecValTok{1}\SpecialCharTok{{-}}\NormalTok{(n}\SpecialCharTok{/}\NormalTok{N)) }\SpecialCharTok{*}\NormalTok{ (s\_2}\SpecialCharTok{/}\NormalTok{n))}

\FunctionTok{print}\NormalTok{(}\FunctionTok{sprintf}\NormalTok{(}\StringTok{"mu = \%f, B = \%f "}\NormalTok{ , mu, B))}
\end{Highlighting}
\end{Shaded}

\begin{verbatim}
## [1] "mu = 12.500000, B = 7.041250 "
\end{verbatim}

\hypertarget{section-5}{%
\subsection{4.17}\label{section-5}}

\begin{Shaded}
\begin{Highlighting}[]
\NormalTok{tau }\OtherTok{=}\NormalTok{ N }\SpecialCharTok{*}\NormalTok{ y\_bar}

\NormalTok{B }\OtherTok{=} \DecValTok{2}\SpecialCharTok{*}\FunctionTok{sqrt}\NormalTok{((N}\SpecialCharTok{\^{}}\DecValTok{2}\NormalTok{) }\SpecialCharTok{*}\NormalTok{ (}\DecValTok{1}\SpecialCharTok{{-}}\NormalTok{(n}\SpecialCharTok{/}\NormalTok{N)) }\SpecialCharTok{*}\NormalTok{ (s\_2}\SpecialCharTok{/}\NormalTok{n))}

\FunctionTok{print}\NormalTok{(}\FunctionTok{sprintf}\NormalTok{(}\StringTok{"tau = \%f, b = \%f "}\NormalTok{ , tau, B))}
\end{Highlighting}
\end{Shaded}

\begin{verbatim}
## [1] "tau = 125000.000000, b = 70412.498890 "
\end{verbatim}

\hypertarget{section-6}{%
\subsection{4.18}\label{section-6}}

\hypertarget{a.}{%
\subsubsection{a.}\label{a.}}

\begin{Shaded}
\begin{Highlighting}[]
\NormalTok{N }\OtherTok{=} \DecValTok{10000}
\NormalTok{n }\OtherTok{=} \DecValTok{500}

\NormalTok{Deers.mean }\OtherTok{=} \FloatTok{2.30}
\NormalTok{Deers.var }\OtherTok{=} \FloatTok{0.65}

\NormalTok{Rabbits.mean }\OtherTok{=} \FloatTok{4.52}
\NormalTok{Rabbits.var }\OtherTok{=} \FloatTok{0.97}

\NormalTok{mu1 }\OtherTok{=}\NormalTok{ Deers.mean}
\NormalTok{mu2 }\OtherTok{=}\NormalTok{ Rabbits.mean}

\NormalTok{find\_b }\OtherTok{\textless{}{-}} \ControlFlowTok{function}\NormalTok{(n, N, s\_2) \{}
\NormalTok{  B }\OtherTok{\textless{}{-}} \DecValTok{2}\SpecialCharTok{*}\FunctionTok{sqrt}\NormalTok{((}\DecValTok{1}\SpecialCharTok{{-}}\NormalTok{(n}\SpecialCharTok{/}\NormalTok{N)) }\SpecialCharTok{*}\NormalTok{ (s\_2}\SpecialCharTok{/}\NormalTok{n))}
\NormalTok{  B}
\NormalTok{\}}

\NormalTok{b1 }\OtherTok{=} \FunctionTok{find\_b}\NormalTok{(n, N, Deers.var)}
\NormalTok{b2 }\OtherTok{=} \FunctionTok{find\_b}\NormalTok{(n, N, Rabbits.var)}

\FunctionTok{print}\NormalTok{(}\FunctionTok{sprintf}\NormalTok{(}\StringTok{"mu1 = \%f, b1 = \%f , mu2 = \%f, b2 = \%f"}\NormalTok{ , mu1, b1, mu2, b2))}
\end{Highlighting}
\end{Shaded}

\begin{verbatim}
## [1] "mu1 = 2.300000, b1 = 0.070285 , mu2 = 4.520000, b2 = 0.085860"
\end{verbatim}

\hypertarget{b.}{%
\subsubsection{b.}\label{b.}}

\begin{Shaded}
\begin{Highlighting}[]
\NormalTok{find\_v\_bar }\OtherTok{\textless{}{-}} \ControlFlowTok{function}\NormalTok{(b) \{}
\NormalTok{  v\_bar }\OtherTok{=}\NormalTok{ (b}\SpecialCharTok{/}\DecValTok{2}\NormalTok{)}\SpecialCharTok{\^{}}\DecValTok{2}
\NormalTok{\}}

\NormalTok{find\_b\_diff }\OtherTok{\textless{}{-}} \ControlFlowTok{function}\NormalTok{(s1, s2) \{}
\NormalTok{  B }\OtherTok{=} \DecValTok{2} \SpecialCharTok{*} \FunctionTok{sqrt}\NormalTok{(}\FunctionTok{find\_v\_bar}\NormalTok{(b1)) }\SpecialCharTok{+} \FunctionTok{find\_v\_bar}\NormalTok{(b2)}
\NormalTok{  B}
\NormalTok{\}}

\NormalTok{B\_diff }\OtherTok{=} \FunctionTok{find\_b\_diff}\NormalTok{(Deers.var, Rabbits.var)}

\NormalTok{lower\_bound }\OtherTok{=}\NormalTok{ (mu1}\SpecialCharTok{{-}}\NormalTok{mu2) }\SpecialCharTok{{-}}\NormalTok{ B\_diff}
\NormalTok{upper\_bound }\OtherTok{=}\NormalTok{ (mu1}\SpecialCharTok{{-}}\NormalTok{mu2) }\SpecialCharTok{+}\NormalTok{ B\_diff}

\FunctionTok{print}\NormalTok{(}\FunctionTok{sprintf}\NormalTok{(}\StringTok{"b = \%f, [\%f , \%f]"}\NormalTok{ , B\_diff, lower\_bound, upper\_bound))}
\end{Highlighting}
\end{Shaded}

\begin{verbatim}
## [1] "b = 0.072128, [-2.292128 , -2.147872]"
\end{verbatim}

\hypertarget{section-7}{%
\subsection{4.19}\label{section-7}}

\[
\bar{y} = \frac{\sum y_i}{n} \\
s^2 = \frac{\sum(y_i-\bar{y})^2}{n-1}
\]

\begin{Shaded}
\begin{Highlighting}[]
\NormalTok{child }\OtherTok{=} \FunctionTok{c}\NormalTok{(}\DecValTok{1}\SpecialCharTok{:}\DecValTok{10}\NormalTok{)}
\NormalTok{number }\OtherTok{=} \FunctionTok{c}\NormalTok{(}\DecValTok{0}\NormalTok{, }\DecValTok{4}\NormalTok{, }\DecValTok{2}\NormalTok{, }\DecValTok{3}\NormalTok{, }\DecValTok{2}\NormalTok{, }\DecValTok{0}\NormalTok{, }\DecValTok{3}\NormalTok{, }\DecValTok{4}\NormalTok{, }\DecValTok{1}\NormalTok{, }\DecValTok{1}\NormalTok{)}

\NormalTok{N }\OtherTok{=} \DecValTok{1000}
\NormalTok{n }\OtherTok{=} \FunctionTok{length}\NormalTok{(number)}

\NormalTok{y\_bar }\OtherTok{=} \FunctionTok{mean}\NormalTok{(number)}
\NormalTok{s2 }\OtherTok{=} \FunctionTok{var}\NormalTok{(number)}

\NormalTok{mu }\OtherTok{=}\NormalTok{ y\_bar}
\NormalTok{b }\OtherTok{=} \FunctionTok{find\_b}\NormalTok{(n, N, s2)}

\FunctionTok{print}\NormalTok{(}\FunctionTok{sprintf}\NormalTok{(}\StringTok{"mu = \%f, b = \%f"}\NormalTok{ , mu,b))}
\end{Highlighting}
\end{Shaded}

\begin{verbatim}
## [1] "mu = 2.000000, b = 0.938083"
\end{verbatim}

\hypertarget{section-8}{%
\subsection{4.20}\label{section-8}}

\begin{Shaded}
\begin{Highlighting}[]
\NormalTok{N }\OtherTok{=} \DecValTok{99000}
\NormalTok{n }\OtherTok{=} \DecValTok{1000}
\NormalTok{y\_bar }\OtherTok{=} \DecValTok{430} \SpecialCharTok{/}\NormalTok{ n}
\NormalTok{p }\OtherTok{=}\NormalTok{ y\_bar}
\NormalTok{B }\OtherTok{=} \DecValTok{2}\SpecialCharTok{*}\FunctionTok{sqrt}\NormalTok{((}\DecValTok{1}\SpecialCharTok{{-}}\NormalTok{n}\SpecialCharTok{/}\NormalTok{N)}\SpecialCharTok{*}\NormalTok{y\_bar}\SpecialCharTok{*}\NormalTok{(}\DecValTok{1}\SpecialCharTok{{-}}\NormalTok{y\_bar)}\SpecialCharTok{/}\NormalTok{(n}\DecValTok{{-}1}\NormalTok{))}
\FunctionTok{print}\NormalTok{(}\FunctionTok{sprintf}\NormalTok{(}\StringTok{"p = \%f,b = \%f"}\NormalTok{,p, B))}
\end{Highlighting}
\end{Shaded}

\begin{verbatim}
## [1] "p = 0.430000,b = 0.031168"
\end{verbatim}

\hypertarget{section-9}{%
\subsection{4.21}\label{section-9}}

\begin{Shaded}
\begin{Highlighting}[]
\NormalTok{B }\OtherTok{=} \FloatTok{0.02}
\NormalTok{n }\OtherTok{=}\NormalTok{ N}\SpecialCharTok{*}\NormalTok{(y\_bar}\SpecialCharTok{*}\NormalTok{(}\DecValTok{1}\SpecialCharTok{{-}}\NormalTok{y\_bar))}\SpecialCharTok{/}\NormalTok{((N}\DecValTok{{-}1}\NormalTok{) }\SpecialCharTok{*}\NormalTok{B}\SpecialCharTok{\^{}}\DecValTok{2}\SpecialCharTok{/}\DecValTok{4} \SpecialCharTok{+}\NormalTok{ y\_bar}\SpecialCharTok{*}\NormalTok{(}\DecValTok{1}\SpecialCharTok{{-}}\NormalTok{y\_bar))}
\FunctionTok{print}\NormalTok{(}\FunctionTok{sprintf}\NormalTok{(}\StringTok{"n\_0 = \%f"}\NormalTok{, n))}
\end{Highlighting}
\end{Shaded}

\begin{verbatim}
## [1] "n_0 = 2391.808773"
\end{verbatim}

\[
\Rightarrow N \ge \lceil n_0 \rceil = 2392
\]

\end{document}
